\chapter{Project Initiation}

\section{Introduction}

This project develops a comprehensive PlantUML-based diagramming platform combining individual productivity tools with community collaboration features. The web-based solution enables creating, editing, and sharing PlantUML diagrams while fostering collaborative learning environments.

The platform targets developers, software architects, system designers, and educational institutions requiring efficient technical diagram creation and visual documentation tools.

Key objectives include:
\begin{itemize}
    \item Developing user-friendly web interface for PlantUML diagram creation
    \item Implementing robust project and diagram management
    \item Creating collaborative workspace with AI-powered assistance
    \item Building community platform for sharing and discovering diagrams
    \item Ensuring scalable architecture with authentication and administration
\end{itemize}

The project follows agile development using Scrum framework for iterative development and continuous feedback integration.

\section{Requirements Analysis}

\subsection{System Actors}

\textbf{Primary Actors:}
\begin{itemize}
    \item \textbf{User}: Authenticated individuals with full platform access including workspace management and community interaction
\end{itemize}

\textbf{Secondary Actors:}
\begin{itemize}
    \item \textbf{AI System}: Intelligent assistant providing code editing assistance
    \item \textbf{PlantUML Server}: External service for diagram rendering
\end{itemize}

\subsection{Core Requirements}

\subsubsection{Functional Requirements}
\begin{itemize}
    \item \textbf{Authentication}: OAuth via Google/GitHub with cross-device persistence
    \item \textbf{Project Management}: Complete CRUD operations, sharing, and bulk export
    \item \textbf{Workspace}: Interactive editor with real-time rendering and AI assistance
    \item \textbf{Community}: Project exploration, commenting, liking, and forking
    \item \textbf{Profile}: User management and public portfolio display
\end{itemize}

\subsubsection{Non-Functional Requirements}
\begin{itemize}
    \item \textbf{Performance}: Page loads <3s, diagram rendering <5s, 1000 concurrent users
    \item \textbf{Security}: HTTPS/TLS encryption, OAuth 2.0, input validation, vulnerability protection
    \item \textbf{Usability}: Responsive design, WCAG 2.1 Level AA compliance
    \item \textbf{Reliability}: 99.5% uptime, automated backups, graceful error handling
\end{itemize}
\subsection{Product Backlog}

The product backlog represents a prioritized list of features and requirements derived from stakeholder needs and market analysis. Each backlog item follows the user story format and includes priority classification using MoSCoW method (Must have, Should have, Could have, Won't have this time).

\begin{longtable}{|p{0.7cm}|p{3.6cm}|p{0.7cm}|p{9cm}|p{1.5cm}|}
    \caption{Product Backlog with User Stories } \label{tab:product_backlog} \\
    \hline
    \textbf{ID} & \textbf{Feature} & \textbf{Sub-ID} & \textbf{User Story} & \textbf{Priority} \\
    \hline
    \endfirsthead
    
    \multicolumn{5}{c}%
    {{\bfseries \tablename\ \thetable{} -- continued from previous page}} \\
    \hline
    \textbf{ID} & \textbf{Feature} & \textbf{Sub-ID} & \textbf{User Story} & \textbf{Priority} \\
    \hline
    \endhead
    
    \hline \multicolumn{5}{|r|}{{Continued on next page}} \\ \hline
    \endfoot
    
    \hline
    \endlastfoot
    
    \csvreader[no head, late after line=\\]{./backlog1.csv}{}%
    {\csvcoli & \csvcolii & \csvcoliii & \csvcoliv & \csvcolv}
    \end{longtable}
\section{Project Management}

\subsection{Scrum Organization}

\begin{table}[h!]
    \centering
    \begin{tabular}{|l|l|}
        \hline
        \textbf{Role}          & \textbf{Member(s)}             \\ \hline
        Product Owner          & Issam Mekni                   \\ \hline
        Scrum Master           & Issam Mekni                   \\ \hline
        Development Team       & Issam Mekni, Souhaieb Askri   \\ \hline
    \end{tabular}
    \caption{Scrum team roles}
\end{table}


\subsection{Global Use Case Diagram}
\subsection{Sprint Planning}

The project spans 14 weeks across 5 focused sprints:

\begin{table}[h!]
    \centering
    \begin{tabular}{|c|l|c|}
        \hline
        \textbf{Sprint} & \textbf{Focus Area} & \textbf{Duration} \\ \hline
        I & Infrastructure Setup & 2 weeks \\ \hline
        II & Authentication \& Landing Page & 3 weeks \\ \hline
        III & Project Management & 3 weeks \\ \hline
        IV & Diagram \& Project Management & 3 weeks \\ \hline
        V & Community \& Profiles & 3 weeks \\ \hline
    \end{tabular}
    \caption{Sprint planning overview}
\end{table}

\begin{figure}[H]
    \centering
    \includegraphics[width=0.7\textwidth]{./conception/global_use_case_diagram.png}
    \caption{Global Use Case Diagram}
    \label{fig:global_use_case}
\end{figure}

\section{System Architecture}

\subsection{Deployment Overview}

\begin{figure}[H]
    \centering
    \includegraphics[width=0.85\textwidth]{./conception/deployement_diagram.png}
    \caption{Deployment Architecture}
    \label{fig:deployment}
\end{figure}

\subsection{Technology Stack}

\begin{table}[h!]
    \centering
    \begin{tabular}{|p{3cm}|p{8cm}|}
        \hline
        \textbf{Category} & \textbf{Technologies} \\ \hline
        
         Frontend & 
        \includegraphics[width=0.6cm]{pictures/web/logo/next-js.png} Next.js, 
        \includegraphics[width=0.6cm]{pictures/web/logo/reactts-svgrepo-com.png} React, 
        \includegraphics[width=0.6cm]{pictures/web/logo/typescript-official-svgrepo-com.png} TypeScript, \newline
        \includegraphics[width=0.6cm]{pictures/web/logo/tailwind-svgrepo-com.png} Tailwind CSS \\ \hline
        
         Backend & 
        \includegraphics[width=0.6cm]{pictures/web/logo/node-svgrepo-com.png} Node.js, 
        \includegraphics[width=0.6cm]{pictures/web/logo/next-authe.png} NextAuth.js, \newline
        \includegraphics[width=0.6cm]{pictures/web/logo/prisma.png} Prisma ORM \\ \hline
        
         Database & 
        \includegraphics[width=0.6cm]{pictures/web/logo/pgsql-svgrepo-com.png} PostgreSQL \\ \hline
        
         AI Integration & 
        \includegraphics[width=0.6cm]{pictures/web/logo/langchain-icon-seeklogo.png} LangChain \\ \hline
        
         Deployment & 
        \includegraphics[width=0.6cm]{pictures/web/logo/docker.png} Docker, 
        \includegraphics[width=0.6cm]{pictures/web/logo/minio.png} MinIO \\ \hline
        
         Development & 
        \includegraphics[width=0.6cm]{pictures/web/logo/git.png} Git, 
        \includegraphics[width=0.6cm]{pictures/web/logo/github-mark.png} GitHub, 
        \includegraphics[width=0.6cm]{pictures/web/logo/vscodium-icon.png} VSCodium, \newline
        \includegraphics[width=0.6cm]{pictures/web/logo/linux.png} Linux \\ \hline
    \end{tabular}
    \caption{Core technology stack with icons}
\end{table}

\section{Conclusion}

The project initiation phase successfully established a comprehensive foundation through systematic requirement analysis, stakeholder identification, and strategic Scrum-based planning. The structured approach ensures focused development on core functionality while maintaining flexibility for future enhancements.

Key achievements include clear actor identification, comprehensive requirement specification, prioritized product backlog, realistic sprint planning, and established project management framework. This foundation positions the project for successful progression through technical architecture design and implementation phases.