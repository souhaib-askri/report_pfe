\chapter[Overview]{Overview of the Project}

\section{Introduction}

This chapter introduces the host organization Goodwill Engineering and presents the project context, including the problem statement and proposed AI-driven UML generation solution.
It also outlines the Agile methodology used for development and concludes with the project's expected outcomes.
\section{Overview of the Host Organization}
Goodwill Engineering is a Tunisian company specialized in developing software solutions. It was founded in 2011 by consultants with over 10 years of experience in SAGE solutions. As a growing IT services and engineering company (SSII), Goodwill is the official and authorized distributor of SAGE solutions in Tunisia, making it a key partner for many small and medium-sized enterprises. The company has extensive expertise in building customized systems for the finance and insurance sectors, with a strong focus on Business Intelligence, Big Data, analytical accounting, and payroll management. Goodwill also provides integrated solutions in system integration and ERP systems, aiming to support digital transformation and enhance operational efficiency and decision-making within organizations.
\section{Presentation of the Project Context}

\subsection{Problem Statement}

UML creation using GUI and textual tools presents several challenges:

\textbf{GUI-Based Tools:} Difficult to master, time-consuming, limited collaboration, and weak version control integration.

\textbf{Textual Tools:} Require syntax knowledge (e.g., PlantUML, Mermaid), lack real-time feedback, and pose debugging difficulties.

\textbf{Integration Issues:} Poor workflow integration and limited automation.

\subsection{Existing Solutions}

\textbf{AI-Based Tools:} Use LLM to generate diagrams from user input, but often lack accuracy.Tools like ChatUML\cite{1} and DiagrammingAi\cite{2} provide fast feedback but lack support for complex cases.

\textbf{Limitations:} Existing tools lack full AI integration, offer inconsistent quality, and miss collaborative/community features.

\subsection{Proposed Solution}

\textbf{Core Idea:} The platform combines LLM with PlantUML to interpret natural language and generate accurate diagrams.

\textbf{Key Features:} Real-time validation, collaborative editing, version control support, and a marketplace for sharing templates.

\textbf{Architecture:} Microservices separate LLM, generation, and UI layers for scalability.

\textbf{Advantages:} Professional-quality output, community-oriented design, and user-friendly interfaces.


\section{Methodology Agile and Scrum Framework}

Agile promotes iterative development and adaptability, ideal for evolving AI projects. Scrum enhances Agile through defined roles (Product Owner, Scrum Master, Development Team), events (Planning, Daily, Review, Retrospective), and artifacts (Product Backlog, Sprint Backlog, Increment).

This framework ensures regular inspection, collaboration, and adaptation, supporting continuous improvement throughout the development process.


\section{Conclusion}
This project addresses key limitations in UML generation by integrating LLM with PlantUML.
The proposed platform enhances usability, collaboration, and automation through a scalable architecture.
By leveraging AI, it democratizes diagramming while maintaining professional quality and precision.
